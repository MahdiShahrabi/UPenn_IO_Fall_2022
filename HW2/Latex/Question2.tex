

\subsection{(a) \& (b) \& (c)}



\begin{table}[h]
\parbox{.45\linewidth}{
    \centering
    \footnotesize{
    {
\def\sym#1{\ifmmode^{#1}\else\(^{#1}\)\fi}
\begin{tabular}{l*{3}{c}}
\hline\hline
            &\multicolumn{1}{c}{(1)}&\multicolumn{1}{c}{(2)}&\multicolumn{1}{c}{(3)}\\
            &\multicolumn{1}{c}{h^}&\multicolumn{1}{c}{P^}&\multicolumn{1}{c}{h^ & P^}\\
\hline
beta2       &                     &                     &                     \\
\_cons      &       0.379\sym{***}&       0.403\sym{***}&       0.384\sym{***}\\
            &    (125.92)         &     (92.99)         &    (133.29)         \\
\hline
beta3       &                     &                     &                     \\
\_cons      &      0.0414\sym{***}&      0.0327\sym{**} &      0.0706\sym{***}\\
            &     (12.56)         &      (3.07)         &     (13.71)         \\
\hline
b1          &                     &                     &                     \\
\_cons      &       1.426\sym{***}&       9.903\sym{***}&      -0.628\sym{***}\\
            &     (71.99)         &    (163.09)         &     (-5.51)         \\
\hline
b2          &                     &                     &                     \\
\_cons      &      -0.131\sym{***}&      -7.394\sym{***}&       0.272\sym{**} \\
            &    (-21.72)         &    (-49.78)         &      (3.03)         \\
\hline
b3          &                     &                     &                     \\
\_cons      &                     &                     &       1.637\sym{***}\\
            &                     &                     &     (54.08)         \\
\hline
b4          &                     &                     &                     \\
\_cons      &                     &                     &      -0.174\sym{***}\\
            &                     &                     &    (-25.64)         \\
\hline
\(N\)       &        1502         &        1502         &        1502         \\
\hline\hline
\multicolumn{4}{l}{\footnotesize \textit{t} statistics in parentheses}\\
\multicolumn{4}{l}{\footnotesize \sym{*} \(p<0.05\), \sym{**} \(p<0.01\), \sym{***} \(p<0.001\)}\\
\end{tabular}
}

    }
    \caption{NLSS}
    \label{tab:my_label}
    }
    \quad
\parbox{.45\linewidth}{
    \centering
    \footnotesize{
    {
\def\sym#1{\ifmmode^{#1}\else\(^{#1}\)\fi}
\begin{tabular}{l*{1}{c}}
\hline\hline
            &\multicolumn{1}{c}{(1)}\\
            &\multicolumn{1}{c}{lemp and dummies}\\
\hline
lemp        &       0.584\sym{***}\\
            &     (44.18)         \\
[1em]
d73         &      -0.169\sym{***}\\
            &     (-7.53)         \\
[1em]
d78         &      -0.153\sym{***}\\
            &     (-7.35)         \\
[1em]
d83         &      -0.220\sym{***}\\
            &    (-10.17)         \\
[1em]
d88         &           0         \\
            &         (.)         \\
[1em]
d357\_73     &      -3.245\sym{***}\\
            &    (-38.88)         \\
[1em]
d357\_78     &      -2.037\sym{***}\\
            &    (-35.87)         \\
[1em]
d357\_83     &      -0.757\sym{***}\\
            &    (-13.16)         \\
[1em]
d357\_88     &       0.408\sym{***}\\
[1em]
\_cons      &       3.661\sym{***}\\
            &     (65.96)         \\
\hline
\(N\)       &        2971         \\
\hline\hline
\multicolumn{2}{l}{\footnotesize \textit{t} statistics in parentheses}\\
\multicolumn{2}{l}{\footnotesize \sym{*} \(p<0.05\), \sym{**} \(p<0.01\), \sym{***} \(p<0.001\)}\\
\end{tabular}
}

    }
    \caption{For lemp and Dummies}
    \label{tab:my_label}
}

\end{table}

\subsection{(d)}
\
As we can see from Table 2, the results do not change much between specifications. This suggests that the additional information brought from exit decisions can substitute for the usual inversion of the investment decision function. Moreover, the results in Tables 2 and 3 are reasonable: the coefficient on labor is 0.58, and the coefficient on capital is around 0.38. These results are very similar to the results from specification (3) from Problem 1 — the specification that modeled transmitted shocks as AR processes. We can conclude that allowing for a more general process than AR does not add much. And as we have seen in problem 1, allowing for fixed effects even creates additional problems.


